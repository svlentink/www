%compile with writelatex.com
\documentclass[a4paper]{article}
\usepackage[english]{babel}
\usepackage[utf8]{inputenc}
\usepackage{amsmath}
\usepackage{graphicx}
\usepackage[colorinlistoftodos]{todonotes}
\usepackage{color}
\usepackage{hyperref}
\usepackage{float}
\usepackage{fancyhdr}
\usepackage[yyyymmdd,hhmmss]{datetime}%for generating auto time stemp
%from geogebra
\usepackage{pstricks-add}
\usepackage{pgf,tikz}
\usetikzlibrary{arrows}
\newrgbcolor{xdxdff}{0.49 0.49 1}
\newrgbcolor{uququq}{0.25 0.25 0.25}
\newrgbcolor{vvvvvv}{0.33 0.33 0.33}
\newrgbcolor{ayayay}{0.66 0.66 0.66}
\newrgbcolor{srsrsr}{0.13 0.13 0.13}
\definecolor{qqwuqq}{rgb}{0.13,0.13,0.13}
\definecolor{uququq}{rgb}{0.25,0.25,0.25}
\definecolor{xdxdff}{rgb}{0.66,0.66,0.66}
\definecolor{qqqqff}{rgb}{0.33,0.33,0.33}
\newcommand{\degre}{\ensuremath{^\circ}}
\psset{xunit=1.0cm,yunit=1.0cm,algebraic=true,dotstyle=o,dotsize=3pt 0,linewidth=0.8pt,arrowsize=3pt 2,arrowinset=0.25}
%end from geogebra
\renewcommand{\deg}{\ensuremath{^{\circ}}}%enable degree symbol for angles and celcius
\pagestyle{fancy}
\newcommand{é}{\'e}
\newcommand{ë}{\"e}
%edit this if you need a different date
\newcommand{\todayDate}{\today}
\rfoot{Compiled on \todayDate\ at \currenttime GMT}
\cfoot{}
\lfoot{Page \thepage}


\begin{document}
\begin{titlepage}
    \centering

	\title{Persoonlijke reflectie\\*
		$\in$\\*
		college ascendantberekening\\*
		\small
		$\in$\\*
		colleges\\*
		$\subset$\\*
		Concrete meetkunde}
	\author{\href{http://svlentink.co.nf}{Sander Lentink} \small F131999 \\*
		\small
        	\href{http://wiskuu.nl}{Mathematisch Instituut},
            \href{http://uu.nl}{Universiteit Utrecht},
            the Netherlands}

	\date{\todayDate}

\maketitle

"consumeren van leerstof; [...] 'Concrete meetkunde' biedt,
door de concrete en gemakkelijk toegankelijke aard van de leerstof, de mogelijkheid deze stroom om te keren."
[\href{http://www.jphogendijk.nl/projects/cm/diktaatcm.pdf}{dictaat}
blz 4]
\\[4in]
\scriptsize - In de pdf versie van dit document staan links waar u op kunt klikken. -

\end{titlepage}

\section{Verloop}

Wij hebben dit onderwerp gekozen omdat het ons interessant, nieuw en uitdagend leek.
Dit maakte dat wij enthousiast van start gingen.

Tijdens het vooronderzoek heb ik vaak gelachen, de geschiedenis,
de betekenins van de symbolen etc. maakte dat het leuk was om te lezen.

Ik was begonnen met het uitwerken van de begrippen en een beetje achtergrond informatie opschrijven.
Dit deed ik in een online editor (\href{https://www.writelatex.com}{writelatex})
en had mijn collega de link van het dictaat gestuurd, zodat wij beiden er online in konden werken.

Hij had zijn eigen compiler waar hij mee werkte en had mijn werk ge\"importeerd.
Dit vond ik op het begin lastig, ik had er geen zicht meer op.
Toen ik zag dat hij mijn werk ingekort had begreep ik dat ik beter kon vragen wat hij wilde,
dan zelf aandeel leveren en dat het niet toegevoegd zou worden.
Uit feedback bij andere vakken bleek dat ik soms dingen meer moest loslaten,
dit heb ik dus ook als zeer leerzaam ervaren.
\\
\\
Na vervolgens mijn eigen ascendant uit te hebben gerekend,
begon ik hoofdzakelijk met het toewerken naar de presentatie.
Bij mijn eigen opleiding (informatica) heb ik een presentatie van een beroepsspreker mogen bijwonen over social media,
hij had 160 slides met veel plaatjes die zijn boodschap visueel ondersteunden.
Deze methode van kennisoverdracht heb ik proberen te gebruiken in onze slides.
\\
\\
Tijdens onze samenwerking zijn wij vaak (soms op aandringen van mij) op De Uithof aan het werk gegaan.
Dit werkte voor mij goed; directe communicatie.
\\
\\
Voor de presentatie hadden wij in grote lijnen afgesproken wie wat zou doen en dat we elkaar zouden bijspringen,
mocht iemand iets vergeten.
Tijdens de presenatie, hadden wij opstart problemen.
Waar wij zelf op het begin ook even over gedaan hadden om door te hebben hoe het zit,
landde het bij de groep niet zo snel als wij gehoopt hadden.
Hierin waren wij als niet-wiskundestudenten er van uitgegaan dat zij het sneller zouden oppakken.

Dit leidde tot vertraging en hierdoor moesten wij delen schrappen.
Wat tot gevolg had dat ook de verdeling van wie wat vertelde minder in balans was.
\\
\\
Ook stelde Jan Hogendijk tussendoor de vraag; hoe groot je de hemelbol moet nemen.
Deze vraag zette ons even voor het blok, maar hielp ons en de klas het concept beter te laten landen.
\\
\\
Voor de huiswerkvrijdag had mijn collega de opdrachten aangepast en had ik nog snel een beschrijving voor Truman zijn ascendant en een formulier die de sterrentijd uitrekent online gezet (zodat ze niet telkens de lange formule hoefden over te typen).
Na 10 uur (toen iedereen de tweede vraag gemaakt had) heb ik toen ook die opdracht online gezet zodat ze dat als voorbeeld hadden voor hun eigen sterrentijd.

\newpage

\section{Evaluatie}

\subsection{Feedback studenten}

\begin{itemize}
\item \textbf{Overzicht}; ik lees terug in de feedback dat we meer het grote plaatje hadden moeten schetsen, voordat wij de losse componenten gingen behandelen. Doordat wij begonnen met de basisbegrippen was het niet goed duidelijk waar dit voor diende en was het dus lastig om te volgen.
\item \textbf{Doelgroep}; we hadden onze boodschap correct afgestemd op de doelgroep.
\item \textbf{Aandacht voor de groep}; het is als plezierig ervaren dat wij de tijd namen om onduidelijkheden verder toe te lichten.
\item \textbf{Visuele ondersteuning}; de plaatjes werden door de meeste als een positieve ondersteuning ervaren.
\item \textbf{Concept}; het duurde lang voor dat het gehele concept, met de verschillende samenhangende onderdelen, was geland.
\item \textbf{Presentatie}; \'e\'en individu vond het een overdaad aan dias, de rest had hier geen problemen mee.
\item \textbf{Tempo}; doordat wij veel wilden behandelen gingen wij soms te snel.
\item \textbf{Pedaal}; om niet telkens naar de laptop te reiken hadden wij een voetpedaal, dit werd als "cool" ervaren. Ik hoop niet dat dit afgeleid heeft van de boodschap, maar dat de focus wel op de spreker bleef.
\item \textbf{Spraak}; was goed.
\end{itemize}

\subsection{Wat ging er goed}

\begin{itemize}
\item Doordat wij, tijdens de voorbereidingen, beiden onze focus op een ander onderdeel hadden liepen wij elkaar niet in de weg.
\item De presentatie was voor beiden ten alle tijden toegankelijk (google docs).
\item Tijd nemen, tijdens de presentatie, om onduidelijkheden toe te lichten.
\end{itemize}

\subsection{Wat ging er minder goed}

\begin{itemize}
\item Communicatie; te weinig overleg/doorspreken.
\item Maar één had overzicht over het dictaat.
\item We hebben tijd verspeeld aan een beta versie van Geogebra 3D.
\end{itemize}

\subsection{Verbeterpunten}

\begin{itemize}
\item Meer expliciet communiceren.
\item Beiden ten alle tijden toegang tot het dictaat.
\end{itemize}

\subsection{Tips en tricks}

In dit deel geef ik ervaringen vanuit mijn perspectief voor toekomstige studenten die dit onderwerp willen behandelen.
\\
\\
Wanneer ik kijk naar het
\href{http://www.jphogendijk.nl/projects/cm/diktaatcm.pdf}{uitgereikte dictaat}
zie ik essentiële tips, vooral;
"Denk er wel aan dat twee uur betrekkelijk weinig tijd is voor een presentatie." [blz viii]

Zelf heb ik over deze zin heen gelezen; het is belangrijk om je eigen learning curve te loggen.
Dit kan ondersteuning bieden in het schatten, hoelang het duurt voordat iets landt.
\\
\\
Ecliptic (zodiac), celestial equator en horizon; voor dit onderwerp is het cruciaal dat dit direct helder wordt voor de studenten, evt. met een opdracht.
\\
\\
Voor het uitrekenen van formules met bv. modulo heb ik gekozen voor
\href{http://www.wolframalpha.com/input/?i=%28+%28+%28d+%2B+%28h%2Bm%2F60%29%2F24%29%29+*+%28366.25%2F365.25%29+*+360+%2B+%CE%BB%29+mod+360}{WolframAlpha}.
Samen met een html formulier op de website, zorgde dit ervoor dat niet iedereen telkens alles over hoeft te typen.
Dit werd door sommigen als prettig ervaren.
\\
\\
De laptop van mijn collega had op het begin 2GB RAM, hierdoor was hij minder productief.
Ik had hem extra RAM gegeven en een schroevendraaier.
Dit maakte dat zijn programma's beter werkten en ook dat writelatex (momenteel in mijn browser 300MB)
beter te gebruiken was.

\section{Opbouw presentatie}

Ik heb er voor gekozen om te beginnen met het scherm over statistieken.
Voor veel b\'eta-studenten is het onbekend dat nog relatief veel mensen in horoscopen geloven.
Dit heb ik bewust gedaan, om wat awareness te creëren.
\\
\\
Vervolgens wat achtergrond informatie over de oorspong van het sterrenlezen en over dat vroeger de aarde als middelpunt werd gezien en nu de zon.
\\
\\
Daarna zijn wij begonnen met het introduceren van de belangrijke termen, gerelateerd aan de hemelbol.
Dit hadden wij, samen met de uitleg van sterrentijd voor de pauze gepland.
Echter zou later blijken dat wij dit zwaar onderschat hadden.

De hemelbol hadden wij voor de sterrentijd gepland,
zodat het ruimtelijke plaatje al bekend was en de sterrentijd daarin geplaatst kon worden.
\\
\\
Na de pauze zouden we de ascendant oppakken.
Eerst wat uitleg, om vervolgens een rekenvoorbeeld te doen.
Vervolgens zou ik een voorbeeld ascendant uitwerken op het bord, waarna ze hun eigen konden gaan uitrekenen.
Helaas zijn wij hier nooit aan toe gekomen.
\\
\\
Wanneer er nog tijd over zou zijn, zou ik vervolgens hun persoonlijke klanken kunnen spelen
(aan de hand van hun uitgerekende ascendant).


\newpage

\section{Feedback formulieren}

\subsection{Omgaan met effect, doelgroep, doelstelling en tijd}

\begin{itemize}
\item "Doelgroep was prima, jullie hebben niet alles kunnen vertellen, te weinig tijd"
\item "Prima, structuur ging wat de mist in doordat de begrippen langzaam begrepen werden"
\item "De doelgroep was juist, structuur soms wat warrig. Zeker aan het begin leek het erg gehaast zonder dat er erg veel stof doorheen leek te gaan. Deels door de manier van praten, deels door het razende tempo waarop we door de dias gingen."
\item "Goede opbouw en lekker tempo. Soms was de uitleg niet heel duidelijk en ook niet wat je nu precies aan het doen was."
\item "Soms wat warrige structuur, tempo goed. Doelgroep wel goed ingeschat."
\item "Tijdgebrek door de vele vragen."
\item "Tempo was goed. Niveau was goed te volgen. Leuk dat het zo 'persoonlijk' was"
\item "Leuk onderwerp."
\item "Goed dat jullie steeds wachtte om dingen te verduidelijken. Soms ging dingen wat snel."
\item "Doelgroep adequaat ingeschat?: Jup. Juist tempo?: ging soms snel, spreektempo wel goed"
\item "'Wetenschappers/b\'eta's' en astrologie is geen geschikte combinatie. Lastig om het verhaal dan interessant te maken. Structuur ontbrak nogal."
\end{itemize}

\newpage

\subsection{Omgaan met inhoud}

\begin{itemize}
\item "Vakinhoud klopte wel, Het was niet altijd duidelijk waar jullie het over hadden. Misschien had je de plaatjes wat beter kunnen gebruiken om het onderscheid tussen de verschillende cirkels, termen, etc. duidelijk te maken. Het duurde lang voordat ik het concept snapte."
\item "De klas, inc mijzelf, begreep het pas laat dus daardoor tijd tekort, leuke inhoud, wiskundig niveau zat top."
\item "Soms wat foutjes, en wat chaotisch gebracht waardoor ik het slecht kon volgen."
\item "Vond het interessant onderwerp, goed gebracht."
\item "Af en toe lastig te volgen, maar jullie deden wel je best om het voor iedereen duidelijk te krijgen."
\item "Vakinhoud: correct denk ik, vond moeillijk om in te zien. Helaas weinig doe-opdrachten"
\item "Erg veel vragen, gingen jullie redelijk goed mee om"
\item "Er was net iets te vaak iets onduidelijk. Verder wel mooie p.p."
\item "Leuke plaatjes. Goede bal meegenomen."
\item "Was moeilijk en ging vaak snel"
\item "Volgens mij correct, maar niet altijd even inzichtelijk."
\end{itemize}

\newpage

\subsection{Presentatietechnisch}

\begin{itemize}
\item "Visuele ondersteunig was mooi, maar jullie hadden het meer kunnen gebruiken"
\item "Alles top, visueel was het gewoon erg lastig, maar toch goed geprobeerd"
\item "Teveel dias"
\item "Prima presentatie"
\item "Goed, interressant onderwerp. Start en slot hadden weinig met elkaar te maken?"
\item "Duidelijk gesproken. Veel plaatjes"
\item "Sprak duidelijk, af en toe wat hoog tempo"
\item "Ontspannen houding allebei, maar ik had wel een beetje het gevoel dat de ene Sander meer aan het woord was dan de andere."
\item "Jullie spreken erg duidelijk. De klikker was cool!"
\item "Krullen sander: je zegt vaak "voor de rest", andere sander: je had het over 'draaing van de zon', was voor mij lastig te plaatsen"
\item "Netjes"
\end{itemize}

\subsection{Instrumentele / technische / overige randvoorwaarden}

\begin{itemize}
\item "Zo goed als maar kon"
\item "Sikke powerpoint, die animaties ware heel chill."
\item "Prima"
\item "Prima in orde, leuk dat "pedaal" om door de dia's heen te klikken."
\item "Veel plaatjes, sommige wat rommelig"
\item "Mooie P.P. Goed bord gebruik"
\item "Mooie presentatie."
\item "goed, cool pedaal!"
\item "Duidelijke ondersteuning"
\end{itemize}



\end{document}
