\documentclass[a4paper]{article}
\usepackage[english]{babel}
\usepackage[utf8]{inputenc}
\usepackage{amsmath}
\usepackage{graphicx}
\usepackage[colorinlistoftodos]{todonotes}
\usepackage{amsfonts}
\usepackage{color}
\usepackage{hyperref}
\usepackage{float}
%\usepackage{html}


\begin{document}


	\title{Volvo suspension\\*
		\small
		$\in$\\*
		assignments\\*
		$\subset$\\*
		ModSim}
	\author{Juan Claramunt \small F130712\\*
    	Sander Lentink \small F131999 \\*
		\small
        	\href{http://wiskuu.nl}{Mathematisch Instituut},
            \href{http://uu.nl}{Universiteit Utrecht},
            the Netherlands}

	\date{\today}

\maketitle
\begin{abstract}
This paper looks at the indepentent suspension of a car wheel and the forces applied on it.
\end{abstract}

%\href{http://upload.wikimedia.org/wikipedia/commons/thumb/c/ce/Packard_wishbone_front_suspension_(Autocar_Handbook,_13th_ed,_1935).jpg/414px-Packard_wishbone_front_suspension_(Autocar_Handbook,_13th_ed,_1935).jpg}{front page picture}




\newpage



\section{Introduction}

The course "Modellen en simulaties", taught by prof. dr. ir. J.E. Frank,
provides us with assignments given in the "dictaat" written by F. Beukers.

In this document we look at assignment 4.4.19, about the suspension system of a Volvo.
We will first give a short introduction to the situation.
\\*
\\*
One of the critical points of designing Volvo's new car, the M1503, is the suspension.
Volvo values the safety and comfort of their cars;
with the safety requirement comes the aim that the wheels stay in contact with the road at all times.
The comfort of the cabinet is given by the accelaration of it.
This means, the faster a cabinet accelerates (e.g. taking a speed bumb without slowing down),
the less comfort the passager will experience.

For this new car, the aim is to design a suspension system that maximizes the comfort,
while remaing contact with the road at all times.
\\*
\\*
This new car has independent suspension for each wheel.
This means that every wheel, roughly speaking, caries 25\% of the mass.
The wheels are as follow.

\begin{figure}[H]
\centering
\includegraphics[width=0.25\textwidth]{sistemaSeperatoBasica.png}
\caption{\label{fig:basicConstructionWheel}Basic concept [1]}
\end{figure}
%\href{https://ccrma.stanford.edu/wiki/Images/f/fb/Accelerometer_mass_spring.gif}{link to figure}
%\href{http://ctms.engin.umich.edu/CTMS/Content/Suspension/Simulink/Modeling/figures/susp1.png}{link to figure}
It is a system with two springs which are damped.
On one side by the tire and on the other by the suspension.
The weight of the wheel is approx. 10kg.
The weight of the car, pressing down on a wheel, is between 250 and 400kg (depending on the cargo).


\newpage


\section{Surfaces}

We are going to test our suspension system on three different surfaces.
Each one is selected due to their characteristics.
The three surfaces are represented in two dimensions as a sine wave.
The first surface $u_1(t)$ has an amplitude of 5 cm and a period of 20 meters.
Then, the equation of the sine wave is:

\begin{equation}
u_1(x)=0.05sin(\frac{\pi}{10}x)
\end{equation}

And, as $x=v t+x_0$:

\begin{equation}
u_1(t,x_0)=0.05sin(\frac{\pi}{10}(v t+x_0))
\end{equation}

The function sine is periodic, so the result is independent of $x_0$.

Assuming $x_0=0$ gives us:
\begin{equation}
u_1(t)=0.05sin(\frac{\pi}{10}v t)
\end{equation}

This surface is soft, this is, the height increases (or decreases) 1 cm when the car moves 1 m (average slope 1\%).
Consequently, the suspension can be slowly adapted to the surface.
\\
\\*
The second one $u_2(t)$ has an amplitude of 25 cm and a period of 2 m.
The equation of this surface is:

\begin{equation}
u_2(x)=0.025sin(\pi x) \Rightarrow u_2(t)=0.025sin(\pi v t)
\end{equation}

In this case, the height increases (or decreases) 2.5 cm when the car moves 0.5 m.
The average slope is 5 times bigger than the first surfice (5\%).
\\
\\*
The third surface $u_3(t)$ has an amplitude of 1 cm and a period of 20 cm.
Its correspondent equation is:

\begin{equation}
u_3(x)=0.01sin(10\pi x) \Rightarrow u_3(t)=0.01sin(10\pi v t)
\end{equation}

This surface is very irregular.
The height increases (or decreases) 1 cm when the car moves 2.5 cm.
The average slope is 40\%,
much bigger than the other surfaces average slope.
\\
\\*
The three are very different.
The first one models a large and not so high obstacle
(a regular road with small slope), then, the slope is small.
The second one models several small obstacles in a short space (e.g. a road with some damage).
The third one models many tiny obstacles which are close to each other (a very irregular road).
Using these three surfaces we are modeling very different scenarios.
Furthermore, other surfaces can be modeled as composition of these surfaces.
Every periodic function or continuous function in an interval can be written as
a Fourier serie [2],
then, every surface can be approximated by a "kind of Fourier serie" with the following form:

\begin{equation}
f(t)=\frac{(a_0+b_0+c_0)}{2}\sum_{n=1}^{\infty} a_n u_1(t)+b_n u_2(t)+c_n u_3(t)
\end{equation}


\newpage


\section{Structure of the suspension}
The suspension system is formed by two "blocks" with the form of figure \ref{fig:suspensionbasica}.
Where k is the spring constant,
d the damping constant and x the displacement of the mass.

Our model with both blocks connected is of the form of figure \ref{fig:sistemacompleto}.
Where $m_1$ is 1/4 of the mass of the car, $m_2$ the mass of the wheel,
$x_1$ the displacement of the car, $x_2$ the displacement of the wheel,
u is the surface funcion,
and $k_b$,$k_s$,$d_b$ and $d_s$ the constants for the tire and the suspension.\\



\begin{figure}[H]
\centering
\includegraphics[width=0.4\textwidth]{suspensionbasica.jpg}
\caption{\label{fig:suspensionbasica}Structure of one block[4]}
\end{figure}


\begin{figure}[H]
\centering
\includegraphics[width=0.5\textwidth]{sistemacompleto.png}
\caption{\label{fig:sistemacompleto}Structure of the suspension system [3]}
\end{figure}



\newpage



\section{Mathematical model}

The mathematical model is based on the forces we find in the suspension.
The three forces are:
\begin{enumerate}
	\item The gravitational force.
	\item The damping force.
	\item The spring force.\\
\end{enumerate}

The car is being influenced by the following forces:
\begin{enumerate}
	\item The gravitational force working on the car.
	\item The damping force of the suspension.
	\item The spring force of the suspension.\\
\end{enumerate}

And the wheel is being influenced by:
\begin{enumerate}
	\item The gravitational force working on the wheel.
	\item The damping force of the tire.
	\item The spring force of the tire.
	\item The damping force of the suspension.
	\item The spring force of the suspension.\\
\end{enumerate}

The damping force is proportional to the speed at which the length of the spring will change
(we denote with $d_b$ the constant of the damping force of the tire and $d_s$ the constant of the damping force of the suspension).
The spring force is proportional to the deflection of the spring
(in the same way as before,
we denote with $k_b$ the constant of the spring force of the tire and $k_s$ the constant of the spring force of the suspension).\\


The equations we obtain using the forces are,
related to the mass of the car:
\begin{equation}
	F_1=-d_s(x'_1-x'_2)-k_s(x1-x2)-(g\,m_1)
\end{equation}
As $F=m x''$ by Newton's second law[5]:
\begin{equation}
	m_1 x''_1=-d_s(x'_1-x'_2)-k_s(x1-x2)-(g\, m_1)
\end{equation}
And related to the mass of the wheel:
\begin{equation}
	F_2=d_s(x'_1-x'_2)+k_s(x1-x2)+d_b(u'-x'_2)+k_b(u-x_2)-(g\, m_2)
\end{equation}
\begin{equation}
	m_2 x''_2=d_s(x'_1-x'_2)+k_s(x1-x2)+d_b(u'-x'_2)+k_b(u-x_2)-(g\, m_2)
\end{equation}


\newpage
%https://www.youtube.com/watch?v=O2fU28Pehdc&list=PLcfQmtiAG0X86jeHk7Tbhjqr6HaYLvZ-c


In order to know if the system fulfills the requirements,
we will analyse the stability of the system depending on the spring and demping constants.

Firstly, Volvo give us a test car with the following suspension characteristics:
$$d_b=2000 Ns/m  \,\,\,\,\, d_s=400 Ns/m  \,\,\,\,\,\,k_b=100000 N/m \,\,\,\,\,\,k_s=40000N/m$$

We also know that the car mass (for a single wheel) is between 250 and 400 kg
and the mass of the wheel is 10 kg.
We also assume g=10 m/$s^2$.

We start by looking at the stability of the homogeneous part of the system,
which matrix form is:
\begin{equation}\label{matrix_form}
  \begin{pmatrix} 
    x'_1 \\
    y'_1 \\
    x'_2 \\
    y'_2\\
  \end{pmatrix}
  =
  \begin{pmatrix} 
    0 & 1 & 0 & 0 \\
    \frac{-k_s}{m_1} & \frac{-d_s}{m_1} & \frac{k_s}{m_1} & \frac{d_s}{m_1} \\
    0 & 0 & 0 & 1\\
    \frac{k_s}{m_2} & \frac{d_s}{m_2} & \frac{-k_s-k_b}{m_2}&\frac{-d_s-d_b}{m_2}\\
  \end{pmatrix}
  \begin{pmatrix} 
    x_1 \\
    y_1 \\
    x_2 \\
    y_2\\
  \end{pmatrix}
\end{equation}
\\*
Introducing the values of the constants we obtain:
\begin{equation}
  \begin{pmatrix} 
    x'_1 \\
    y'_1 \\
    x'_2 \\
    y'_2\\
  \end{pmatrix}
  =
  \begin{pmatrix} 
    0 & 1 & 0 & 0 \\
    \frac{-40000}{m_1} & \frac{-400}{m_1} & \frac{40000}{m_1} & \frac{400}{m_1} \\
    0 & 0 & 0 & 1\\
    4000 & 40 & -14000 & -240 \\
  \end{pmatrix}
  \begin{pmatrix} 
    x_1 \\
    y_1 \\
    x_2 \\
    y_2\\
  \end{pmatrix}
\end{equation}
\\*
The conditions to reach an equilibrium are:
\begin{equation}\label{der_x1}
	x'_1(t)=0
\end{equation}
\begin{equation}\label{der_y1}
	y'_1(t)=0
\end{equation}
\begin{equation}\label{der_x2}
	x'_2(t)=0
\end{equation}
\begin{equation}\label{der_y2}
	y'_2(t)=0
\end{equation}
\\*
Then, from \ref{der_x1} and \ref{der_x2} we obtain:
\begin{equation}
	y_1=0 \,\,\& \,\, y_2=0
\end{equation}
\\*
From \ref{der_y1}:
\begin{equation}\label{x1_equals_x2}
	\frac{-k_s}{m_1}x_1+\frac{k_s}{m_1}x_2=0 \Rightarrow x1=x2
\end{equation}
\\*
and from \ref{der_y2} and \ref{x1_equals_x2}:
\begin{equation}
  \frac{k_s}{m_2}x_1+\frac{-k_s-k_b}{m_2}x_2=0
  \Rightarrow
  \frac{k_s}{m_2}x_2+\frac{-k_s-k_b}{m_2}x_2=0
  \Rightarrow
  \frac{-k_b}{m_2}x_2=0
  \Rightarrow
  x_2=0
  \Rightarrow
  x_1=x_2=0 
\end{equation}
(We assume $k_b\neq 0$, which is physically supported,
if not, the spring force is 0).


\newpage


The equilibrium is:
$$x_1=x_2=y_1=y_2=0$$

The result is independent of the values of the constants,
so we can use it in the general case.

To evaluate the stability we use the eigenvalues of the matrix:\\
\\
$$
  \begin{pmatrix} 
	0 & 1 & 0 & 0 \\
    \frac{-40000}{m_1} & \frac{-400}{m_1} & \frac{40000}{m_1} & \frac{400}{m_1} \\
    0 & 0 & 0 & 1\\
    4000 & 40 & -14000 & -240 \\
  \end{pmatrix}
$$
\\
which characteristic polynomial is:
\begin{equation}
  P(\lambda) =
  \frac{400000000}{m_1} +
  \frac{12000000}{m_1}\lambda +
  14000\lambda^2 +
  \frac{120000}{m_1} \lambda^2+240\lambda^3 +
  \frac{400}{m_1}\lambda^3 +
  \lambda^4
\end{equation}
\\
\\
As every coefficient is positive and $P(0)\neq 0$, the eigenvalues are going to be negative.
Let's see it with the roots of $P(\lambda)$:

$$
  \lambda_1 =
  \frac{
  	10(-2(20+7m_1) -
  	\frac{
    	i(-i+\sqrt{3})(400-320m_1+49m_1^2)}
        {(-8000 + 9600 m_1 - 3390 m_1^2-343m_1^3+30\sqrt{15}\sqrt{m_1^3(800-965m_1+343m^2)})^{1/3}}}
  {3m_1}
$$

$$
  \frac{i(i+\sqrt{3})(-8000 + 9600 m_1 - 3390 m_1^2-343m_1^3+30\sqrt{15}\sqrt{m_1^3(800-965m_1+343m^2)})^{1/3})}
  {3m_1}
$$
\\*
\\
\\
Re$(\lambda_1)\approx -140\,\, \forall m_1 \in [250,400]$
\\
\\
$$
  \lambda_2 =
  \frac{
  	20(-20-7m_1 +
  	\frac{
    	400-320m_1+49m_1^2}
        {(-8000 + 9600 m_1 - 3390 m_1^2-343m_1^3+30\sqrt{15}\sqrt{m_1^3(800-965m_1+343m^2)})^{1/3}}}
  {3 m_1}
$$

$$
  \frac{
  	(-8000 + 9600 m_1 - 3390 m_1^2-343m_1^3+30\sqrt{15}\sqrt{m_1^3(800-965m_1+343m^2)})^{1/3})}
    {3m_1}
$$
\\*
\\
\\
Re$(\lambda_2)\in [-0.735,-0.459]\,\, \forall m_1 \in [250,400]$

\newpage

$$
  \lambda_3 =
  	\frac{
    	10(-2(20+7m_1) +
        \frac{
        	i(i+\sqrt{3})(400-320m_1+49m_1^2)}
            {(-8000 + 9600 m_1 - 3390 m_1^2-343m_1^3+30\sqrt{15}\sqrt{m_1^3(800-965m_1+343m^2)})^{1/3}}}
    {3m_1}
$$

$$
  -\frac{(1+i\sqrt{3})(-8000 + 9600 m_1 - 3390 m_1^2-343m_1^3+30\sqrt{15}\sqrt{m_1^3(800-965m_1+343m^2)})^{1/3})}
  {3m_1}
$$
\\*
\\
\\
Re$(\lambda_3)\in [-0.735,-0.459]\,\, \forall m_1 \in [250,400]$
\\
\\
$$
	\lambda_4=-100 \,\, \forall m_1 \in [250,400]
$$
[solutions obtained using Mathematica]\\

As the real part of each eigenvalue is less than 0,
then, the homogeneous part is stable.[6]






The other factor that can cause unstability is the not homogeneous part.
First, we find a specific solution for our system.

We look for a particular solution of the form:
$$x_1(t)=A sin(\omega t)+B cos(\omega t)+C$$
\begin{equation}
	x_2(t)=D sin(\omega t)+E cos(\omega t)+F
\end{equation}

Using the computational software program Mathematica,
we obtain the solutions for the constants A,B,C,D,E,F depending on the speed, v.
The solutions (in this case the road $u_3(t)$,
and the mass of the car set to 400 kg) are shown as graphs:

A:
\begin{figure}[H]
\centering
\includegraphics[width=0.7\textwidth]{graficaA.jpg}
\caption{\label{fig:valOfA}Value of A depending on v}
\end{figure}

\newpage

B:
\begin{figure}[H]
\centering
\includegraphics[width=0.7\textwidth]{graficaB.jpg}
\caption{\label{fig:valOfB}Value of B depending on v}
\end{figure}

$C=-0.141$
\\
\\*
D:
\begin{figure}[H]
\centering
\includegraphics[width=0.9\textwidth]{graficaC.jpg}
\caption{\label{fig:valOfC}Value of C depending on v}
\end{figure}

\newpage

E:
\begin{figure}[H]
\centering
\includegraphics[width=0.9\textwidth]{graficaD.jpg}
\caption{\label{fig:valOfD}Value of D depending on v}
\end{figure}
$F=-0.041$
\\
\\*
As we see in the graphs if we choose $v\in[0,200]$,
the contants $A,B,E\rightarrow \infty$ and $D\rightarrow -\infty$,
then the solution is unstable.
Therefore, the suspension system of the test car does not satisty the conditions of our problem.\\



\newpage



As the test car is not stable;
we are going to determine new values for the parameters $d_b$, $d_s$, $k_b$ and $k_s$ satisfying the conditions.

The values we propose are:
\begin{equation}
	d_b=4000 Ns/m
\end{equation}
\begin{equation}
	d_s=2000 Ns/m
\end{equation}
\begin{equation}
	k_b=80000 N/m
\end{equation}
\begin{equation}
	k_s=30000 N/m
\end{equation}
\\*
Let's look at these values fulfilling the requirements.

First, we will see that the homogeneous solutions are stable using the eigenvalues
of the matrix in \ref{matrix_form} with the new values.
\\
\\*
The characteristic polynomial is\\
$
  P(\lambda) =
  \frac{240000000}{m_1} +
  \frac{28000000}{m_1} \lambda +
  11000\lambda^2 +
  \frac{830000}{m_1} \lambda^2 +
  600 \lambda^3 +
  \frac{2000}{m_1} \lambda^3 +
  \lambda^4
$
\\
\\*
Then, the real part of the eigenvalues are:
$$(\lambda_1)\in[-18.8521,-18.8032]\,\, \forall m \in [250,400]$$
\\*
Re$(\lambda_2)\in [-2.70107,-1.69044]\,\, \forall m \in [250,400]$

$$\lambda_3= $$ %\textcolor{red}{lambda 3}
\\*
Re$(\lambda_3)\in [-2.70107,-1.69044]\,\, \forall m \in [250,400]$

\begin{equation}
	\lambda_4\in [-583.795,-582.767] \,\, \forall m \in [250,400]
\end{equation}
[solutions obtained using Mathematica]
\\
\\*
As the real part of each eigenvalue is less than 0, then, the homogeneous part is stable.[6]
\\
\\*
Now, let's see that the not homogeneous part is also stable:
\\*
We look for a particular solution of the form:
$$x_1(t)=A sin(\omega t)+B cos(\omega t)+C$$
\begin{equation}
	x_2(t)=D sin(\omega t)+E cos(\omega t)+F
\end{equation}



\newpage



In the case of the surface $u_1(t)$ the graphs of the variables are:
\begin{figure}[H]
\centering
\includegraphics[width=0.9\textwidth]{graficaconstantespientre10.jpg}
\caption{\label{fig:3d_u1_400}Values of A,B,D,E depending on v and t ($m_1=400$)}
\end{figure}

\begin{figure}[H]
\centering
\includegraphics[width=0.9\textwidth]{graficaconstantespientre102.png}
\caption{\label{fig:3d_u1_250}Values of A,B,D,E depending on v and t ($m_1=250$)}
\end{figure}


\newpage


In the case of the surface $u_2(t)$ the graphs of the variables are:

\begin{figure}[H]
\centering
\includegraphics[width=0.9\textwidth]{graficaconstantespi.jpg}
\caption{\label{fig:3d_u2_400}Values of A,B,D,E depending on v and t ($m_1=400$)}
\end{figure}

\begin{figure}[H]
\centering
\includegraphics[width=0.9\textwidth]{graficaconstantespi2.jpg}
\caption{\label{fig:3d_u2_250}Values of A,B,D,E depending on v and t ($m_1=250$)}
\end{figure}


\newpage



And finally, in the case of the surface $u_3(t)$ the graphs of the variables are:

\begin{figure}[H]
\centering
\includegraphics[width=0.9\textwidth]{graficaconstantespipor10.jpg}
\caption{\label{fig:3d_u3_400}Values of A,B,D,E depending on v and t ($m_1=400$)}
\end{figure}

\begin{figure}[H]
\centering
\includegraphics[width=0.9\textwidth]{graficaconstantespipor102.jpg}
\caption{\label{fig:3d_u3_250}Values of A,B,D,E depending on v and t ($m_1=250$)}
\end{figure}


\newpage


With the new values of $k_b$,$k_s$,$d_b$ and $d_s$,
the values of the constants A,B,D,E do not go to$ \pm \infty$ for $v\in [0,200]$, $t\in [0,2\pi]$.
\\
$$C=\frac{-10 m_1}{k_s}+F<\infty$$
$$F=\frac{-10\frac{-10 m_1}{m_2}}{k_b}<\infty$$
\\
Then, the suspension system with the last values of $k_b$,$k_s$,$d_b$ and $d_s$ is stable
and satisfies the requirements of the problem. 



\newpage



\section{References}

The references are shown following alphabetical order:\\

\begin{itemize}
	\item [1]Beukers, F. Dictaat "Modellen en simulaties"
	\item [2]Fernandez, L.A. Introducción a las ecuaciones en derivadas parciales. Universidad de Cantabria.2013.
    \item [3]Florin, A.,Ioan, M.,Liliana P. Pasive suspension modeling using matlab, quarter car model, imput signal step type.  
    \item [4]Milliken, W. F., Milliken, D. L.1995 Race Car vehicle dynamics
	\item [5]Newton, I.Principia mathematica.
	\item [6]Sleijpen G. Slides chapter 4 'Modellen en simulatie'.

\end{itemize}



\end{document}
