\documentclass[a4paper]{article}

\usepackage[english]{babel}
\usepackage[utf8]{inputenc}
\usepackage{amsmath}
\usepackage{graphicx}
\usepackage[colorinlistoftodos]{todonotes}
\usepackage{amsfonts}
\usepackage{color}


\begin{document}


	\title{Bouncing ball\\*
		\small
		$\in$\\*
		assignments\\*
		$\subset$\\*
		ModSim}
	\author{Juan Claramunt \small F130712\\*
    	Sander Lentink \small F131999 \\*
		\small Mathematisch Instituut, Universiteit Utrecht, the Netherlands}

	\date{\today}

\maketitle
\begin{abstract}
In this paper we are going to study the behaviour of a ball bouncing on a table. In the first model, the table is static and in the second model, the table moves perpendicular to the ball flight.
\end{abstract}



\newpage



\section{Introduction}

The course "Modellen en simulaties", taught by prof. dr. ir. J.E. Frank, provides us with assignments given in the "dictaat" written by F. Beukers.

In this document we look at assignment 3.4.6, about a ball bouncing on a table. We will first give an short introduction to the situation.
\\*
\\*
We let a ball drop on a table, from which it bounces back. The given formula for the time to the subsequent impact:
\begin{equation}
\tau=\frac{2v}{g}
\end{equation}
with $g$ being the gravitational acceleration (the $2$ stands for the up and down movement).
The speed will gradually decrease during its flight upwards until it reaches the upper level, when all kinetic energy is transformed to potential energy and the ball falls downwards again.
The next impact will transform the potential energy back to kinetic energy again.
\\*
\\*
When we look at the $n$-th impact, the subsequent,
($n$+1)th impact is given by the time step $t_{n+1} = t_n + \tau$.
Here the index $n$ states that the time in between $\tau$,
is dependent on the initial velocity $v_n$
and the new starting velocity $v_{n+1} = \alpha v_n$.
For the enery loss during flight and bouncing back we take $\alpha \in [0,1]$.
Air resistance is disregarded. When $\alpha = 1$ the ball doesn't lose any energy, when $\alpha = 0$ the ball won't bounce at all. [1]





\newpage



% assignment 1, opdracht 1

\section{First model}
When we use a gravitational acceleration $g=2$, the time to the next impact becomes $\tau=v$.
%\textcolor{red}{%
%As g=2 in those units, the time untill next impact $\tau=v$.}%
Then, $t_{n+1}=t_n+v_n$ and $v_{n+1}=\alpha v_n$.
%\textcolor{red}{%
%where $\alpha$ is a constant that measure the loss of energy.($\alpha=1$ there is no loss of energy, $\alpha=0$ the ball lose all the energy.As these cases are not very realistic we suppose $0\leq \alpha \leq 1$ ).}%
For the behavior of the ball we are interested in a bouncing effect.
As we can learn from the introduction we need $0\leq \alpha \leq 1$ for this to happen.
So we have:
\begin{equation}
f_{\alpha}:\begin{pmatrix} 
t\\
v\\
\end{pmatrix}=\begin{pmatrix} 
t+v\\
\alpha v\\
\end{pmatrix}
\end{equation}
\label{sec:examples}
We see that (t,v) is a fixed point when
%\textcolor{red}{
%Which points (t,v) are fixed points?
%(t,v) is a fixed point if}
$(t_{n+1},v_{n+1})=f_\alpha(t_n,v_n)=(t_n,v_n)$.
First we have to solve the following system of equations:
\begin{equation}\label{t=t+v}
t=t+v
\end{equation}
\begin{equation}
v=\alpha v
\end{equation}

From equation \ref{t=t+v} we obtain:
\begin{equation}
v=0
\end{equation}
Then, the fixed points are the points (t,0) $\forall t$.
This means that the fixed points are the points with velocity 0, and in this case,
as $\tau=v$, there are the moments of impact.
%\textcolor{red}{
%Now, t}
To compute the stability of the fixed poits we need to obtain the Jacobian matrix of $f_\alpha$:
\begin{equation}
Df(t,0)=\begin{pmatrix} 
1 & 1\\
0 & \alpha \\
\end{pmatrix}
\end {equation}
The eigenvalues of this matrix are clearly $\lambda_1=1$ and $\lambda_2=\alpha$. As $\alpha<1$ and $\lambda_1=1$ we have no information [3].
Then, we choose a point (t,v) close to the fixed point (t,0), the point ($t_0,v_0$), such that:
$v_0 \approx 0 , v_0\neq 0,t_0\approx t$. Applying $f_\alpha$ resursively we obtain:
\begin{equation}
t_1=t_0+v_0
v_1=\alpha v_0 
\end{equation}
\begin{equation}
t_2=t_1+v_1=t_0+v_0+\alpha v_0
v_2=\alpha v_1=\alpha^2 v_0
\end{equation}
\begin{equation}
t_n=t_0+v_0+\alpha v0+\alpha^2 v_0+...+\alpha^{n-1} v_0=t_0+v_0(1+\alpha+...+\alpha^{n-1})
v_n=\alpha^n v_0
\end{equation}
If we compute the limit when $n\rightarrow \infty$, $v_n\rightarrow 0 \, \forall(t_0,v_0)$,
\\*
however, $t_n\rightarrow t \Leftrightarrow \lim_{n \rightarrow \infty} (t_0+v_0(1+\alpha+...+\alpha^{n-1}))=t$.
\\*
Then, $(t_n,v_n)\rightarrow (t,0)$ if $(t_0,v_0)$ satisfy: $\lim_{n \rightarrow \infty} (t_0+v_0(1+\alpha+...+\alpha^{n-1}))=t$, this is $t=t_0+v_0\frac{1}{1-\alpha}$.
As not every $(t_0,v_0 )\approx (t,0)$ satisfies it, the fixed points are not stable. 




\newpage



% assignment 2, opdracht 2

\section{Second model}

For the next situation we look at a table that is propelled periodically.
The table top stays perpendicular to the vertical shaft,
but moves up and down according to $-\beta sin(t)$.
For the time unit we use a period in which the table vibration is set to $2\pi$.
We asume $\beta << v$.
\\*
The moving table top causes the distance to increase or decrease,
which in turn has effect on the time it takes for the ball to bounce back.
In this situation that effect is negligible, and isn't used.


$t_{n+1}$ depends on $t_n$ and $v_n$ as in the first model: 
\begin{equation}\label{secMod_first}
t_{n+1}=t_n + v_n
\end{equation}
The movement of the ball is the same, then $v_{n+1}^{ball}=\alpha v_n^{ball}$
The table moves according to $x=-\beta sin(t)$, then, the velocity is $v^{table}=x'=-\beta cos(t)$, and as t depends on t and v we obtain $v^{table}=-\beta cos(t+v)$. If we compose both movements and add a term caused by the lost of energy ($\alpha v^{table}$), we obtain:
\begin{equation}\label{secMod_second}
v_{n+1}=\alpha v_n -\alpha \beta cos(t_n+v_n) -\beta cos(t_n+v_n)=\alpha v_n-(1+\alpha)\beta cos(t_n+v_n)
\end{equation}
We define $\gamma=(1+\alpha)\beta$.\\
Using equations \ref{secMod_first} and \ref{secMod_second}, we obtain:

\begin{equation}
f_{\alpha,\gamma}:\begin{pmatrix} 
t\\
v\\
\end{pmatrix}=\begin{pmatrix} 
t+v\\
\alpha v-\gamma cos(t_n+v_n)\\
\end{pmatrix}
\end{equation}

Again, $0\leq \alpha \leq 1$ (as we are interested in a realistic situation), and as $\beta$ is the amplitude so $\beta>0$, $\gamma>0$.
\\
\\
Now, we wonder if $f_{\alpha,\gamma}$ is invertible.\
First, we compute the Jacobian determinant:
\begin{equation}
Df(t,v)=\left|\begin{pmatrix}
1 & 1\\
\gamma sin(t+v) & \alpha + \gamma sin(t+v) \\
\end{pmatrix}\right|=\alpha + \gamma sin(t+v) - \gamma sin(t+v) =\alpha
\end{equation}

As the determinant is a constant (in this case $0 \leq \alpha \leq 1$) we can chose a value for the parameters such that f is invertible. It is ensured by the inverse function theorem.The theorem is: Let $f:A\subseteq \mathbb{R}^n \rightarrow \mathbb{R}^n$ be a $C^1$ function, $a\in A$, as $Df(a)\neq 0$ and f(a)=b, $\exists \, U,V \subset \mathbb{R}^n$ open sets such that $a\in U$, $b\in V$ and $f:U\rightarrow V$ is a bijection, $f^{-1}$ exists and $f^{-1}:V\rightarrow U$ is $C^1$. Our function satisfies the conditions of the theorem, then, the inverse of $f_{\alpha,\gamma}$ exists. [1]




\newpage



In order to find the inverse function, we propose the following system of equations:
\begin{equation}\label{x=t+v}
x=t+v
\end{equation}
\begin{equation}\label{secMod_y=alpha}
y=\alpha v - \gamma cos(t+v)
\end{equation}
Now, we isolate the variables t and v:\\
From equation \ref{x=t+v}:
\begin{equation}\label{x=t-v}
t=x-v
\end{equation}
From equations \ref{x=t+v} and \ref{secMod_y=alpha}:
\begin{equation}
y=\alpha v - \gamma cos(x) \Rightarrow v=\frac{y+\gamma cos(x)}{\alpha}
\end{equation}
And back in equation \ref{x=t-v}:
\begin{equation}
t=x-\frac{y+\gamma cos(x)}{\alpha}
\end{equation}

Then, the inverse function is:
$f^{-1}(x,y)=(x-\frac{y+\gamma cos(x)}{\alpha},\frac{y+\gamma cos(x)}{\alpha})$.

Now, we introduce a new function $\psi_n$ to define a coordinate transformation.\\
For all $n\in \mathbb{Z}$:
$$
\psi_n:\mathbb{R}^2\longrightarrow \mathbb{R}^2\\
$$
\begin{equation}
(t,v)\mapsto (t+2\pi n,v)
\end{equation}

The composition of $f_{\alpha,\gamma}$ and $\psi_n$ is commutative, this is,
\\*
$f_{\alpha,\gamma}(\psi_n(t,v))=\psi_n(f_{\alpha,\gamma}(t,v))$.
Let's prove it:
\begin{equation}
f_{\alpha,\gamma}(\psi_n(t,v))=(t+v+2\pi n,\alpha v-cos(t+v+2\pi n))
\end{equation}
\begin{equation}
\psi_n(f_{\alpha,\gamma}(t,v))=(t+v+2\pi n,\alpha v-cos(t+v))
\end{equation}
\\
Then:
$$
f_{\alpha,\gamma}(\psi_n(t,v))=\psi_n(f_{\alpha,\gamma}(t,v)) \Longleftrightarrow$$
$$f_{\alpha,\gamma}(\psi_n(t,v))_1=\psi_n(f_{\alpha,\gamma}(t,v))_1 \,\& \,
f_{\alpha,\gamma}(\psi_n(t,v))_2=\psi_n(f_{\alpha,\gamma}(t,v))_2
$$
$$
\Longleftrightarrow t+v+2\pi n=t+v+2\pi n \, \& \, \alpha v-cos(t+v+2\pi n) = \alpha v-cos(t+v)
$$   
It is clear that $t+v+2\pi n=t+v+2\pi n$.\\
Let's see that $\alpha v-cos(t+v+2\pi n) = \alpha v-cos(t+v)$.
\\As the funtion cos(x) is $2\pi$-periodic, $cos(x)=cos(x+2\pi n) \, \forall n\in \mathbb{Z}$,
\\then, $\alpha v-cos(t+v+2\pi n) = \alpha v-cos(t+v) \, \forall n\in \mathbb{Z}$.

Then, the result is independent of the order of the composition.
Then, adding $2\pi n$ to the first component, before or after applying $f_{\alpha,\gamma}$,
does not affect the result.
Then, as the first component is independent of $2\pi n$, we can work with the first coordinate in $[0,2\pi)$.




\newpage




Next, we are wondering if the redefined function is also invertible.
This means we have to compute the Jacobian matrix.
As the difference between both functions only depends on a constant term ($2\pi n$,
as in the redefined funcion t lies in $[0,2\pi)$,
the partial derivatives of both functions are going to be the same, and the Jacobian matrix is:

\begin{equation}\label{secMod_dfRedefined}
Df_{redefined}:\begin{pmatrix} 
1 & 1\\
\gamma sin(t+v) & \alpha + \gamma sin(t+v)\\
\end{pmatrix}=Df_{\alpha,\gamma}
\end{equation}

The jacobian determinant is again $\alpha$, so,
by the inverse function theorem we are sure that the inverse of the redefined function exists.
We have the system:
\begin{equation}\label{x=modPi}
x=(t+v)mod2\pi
\end{equation}
\begin{equation}\label{secMod_y=alphaV-gamm}
y=\alpha v -\gamma cos(t+v)
\end{equation}
We proceed by isolating t and v as functions of x and y (t(x,y),v(x,y)).
\\*
From equation \ref{secMod_y=alphaV-gamm}:
\begin{equation}\label{secMod_alphaV=y+}
\alpha v=y +\gamma cos(t+v)\Rightarrow v=\frac{y +\gamma cos(t+v)}{\alpha}
\end{equation}
From equations \ref{x=modPi} and \ref{secMod_alphaV=y+}
\begin{equation}
x=(t+v)mod2\pi \Rightarrow x=t+v+2\pi n
\end{equation}
With $n \in \mathbb{Z}$ such that $(t+v+2\pi n) \in [0,2\pi)$
Then:
\begin{equation}
x=t+v+2\pi n \Rightarrow t=x-v-2\pi n ,\,n \in \mathbb{Z}|(t+v+2\pi n) \in [0,2\pi)
\end{equation}

To find which points (t,v) are fixed points,
we remember that (t,v) is a fixed point if $(t_{n+1},v_{n+1})=f(t_n,v_n)=(t_n,v_n)$.
Written as equations: 
\begin{equation}\label{t=modPi}
t=(t+v)mod2\pi
\end{equation}
\begin{equation}\label{secMod_v=alphaV-gamm}
v=\alpha v -\gamma cos(t+v)
\end{equation}
From equation \ref{t=modPi}:
\begin{equation}\label{secMod_t=t+v+2PIn}
t=t+v+2\pi n ,\,n \in \mathbb{Z}|(t+v+2\pi n) \in [0,2\pi) \Rightarrow v=-2\pi n
\end{equation}
And from equations \ref{secMod_v=alphaV-gamm} and \ref{secMod_t=t+v+2PIn}:
$$
-2\pi n=\alpha (-2\pi n) -\gamma cos(t-2\pi n)\Rightarrow -2\pi n(1-\alpha)=-\gamma cos(t)$$
\begin{equation}
\Rightarrow cos(t)=\frac{2\pi n(1-\alpha)}{\gamma}\Rightarrow t=arccos(\frac{2\pi n(1-\alpha)}{\gamma})
\end{equation}




\newpage




The "possible" fixed points with $v<0$ (n positive) make no sense when $\tau=v$ and $\tau$ is the time to next impact, because time cannot be negative.
The fixed points when $v=0$ are the points when the ball is impacting.
If $v=0 \, \forall t$ the ball bounces in an infinitesimal way,
we are not interested in that case (similar to $\alpha=0$ case).
\\
Next we will look at, if the points are stable or not?

First we compute the eigenvalues of $Df_{redefined}$.
The fixed points are stable $\Leftrightarrow \lambda_2<\lambda_1<1 \Leftrightarrow -2(1+\alpha)<\mu<0$,
where $\mu=\gamma sin(t)$.
\\First:
\begin{equation}\label{secMod_mu_lessThan_zero}
\mu<0\Leftrightarrow \gamma sin(t)<0, (\gamma>0) \Leftrightarrow sin(t)<0 \Leftrightarrow 0<t<\pi
\end{equation}
$$-2(1+\alpha)<\mu \Leftrightarrow -2(1+\alpha)<\gamma sin(t) \Leftrightarrow -2(1+\alpha)<(1+\alpha)\beta sin(t) $$
\begin{equation}
\Leftrightarrow -2<\beta sin(t) \Leftrightarrow
\left\{
\begin{array}{cl}
\beta>\frac{-2}{sin(t)}&\mbox{si } 0 < t < \pi \\
\beta<\frac{-2}{sin(t)}&\mbox{si } \pi < t < 2\pi \\
-2<0 &\mbox{si } t=0,\pi
\end{array}\right.
\end{equation}
Then, from equation \ref{secMod_mu_lessThan_zero},
$0<t<\pi \Rightarrow \beta>\frac{-2}{sin(t)}$ and as $\beta>0$ it is always true.
Therefore, the fixed points are stable if $0<t<\pi$.

For well chosen parameters we can find a stable 2-period orbit.
Then, if we increase the value of $\gamma$, this orbit branch of in an stable orbit of period 4, then in and stable orbit of period 8 and so on.

As in the logistic case, the period doubling leads to chaos.
In order to show this behaviour we are going to use the Smale horseshoe.\\
First, we look for a quedrilateral, Q, such that f(Q) folded in a half lies in Q.
If we take Q=$[1.5,2]x[\frac{1}{10},2\pi+\frac{1}{10}]$ with $\alpha=0.75$ and $\gamma=0.125 \Rightarrow \beta=\frac{0.125}{1.75}<<v$,
we obtain the quadrilateral f(Q) determined by the points (1.5,-0.0088),(1.5,4.7035), (2,0.0520), (2,4.7644).
If we would fold it in half,the new quadrilateral $f(Q)_{folded}$ is determined by the points (1.75,0.0216), (1.75, 4.7340), (2,0.0520) and (2,4.7644).\\
As f is continuous in (at least) $[0,2\pi)$x$\mathbb{R}$, $f(Q), Q \subset [0,2\pi)$x$\mathbb{R}$ and (1.75,0.0216),\\(1.75, 4.7340), (2,0.0520), (2,4.7644)$\in Q$, then  $f(Q)_{folded}\subset Q$.\\ 
Now, we are interested in the intersection over k ($k\in \mathbb{Z}$) of $f^k(Q)$, this is $\Lambda =\underset{k\in \mathbb{Z}}{\bigcap}f^k(Q)$.
We generate the recursion $Q_{n+1}=f^{-1}(Q_n)\cap f(Q_n) \cap Q_n$.
Then, $\Lambda =\underset{k\in \mathbb{Z}}{\bigcap}f^k(Q)=\underset{k\in \mathbb{N}}{\bigcap}Q_k$.
The sequence of sets $(Q_n)_{n\geq0}$ are divided at each step,
so they became smaller, but the amount of points is infinite (the cardinal of the set $Q_n$ is $\aleph_1 \, \forall n$). Then, what we have is a kind of Cantor set. 
Consider the restriction of f to $\Lambda$:
$$ f|_{\Lambda}:\Lambda \longrightarrow \Lambda$$
%The dynamics of this restriction is a significant part of the dynamics of the whole Q at f.
The process $x_{n+1}=f(x_n)$ on $\Lambda$ is chaotic.[3]
For well chosen parameters we can find this chaotic behaviour.
This implies that, for these values the ball bounces in a chaotic way. Then, a small change in the initial values could lead to extremely different results.
For example, we have two close points ($t_1,v_1$) and ($t_2,v_2$) such that the ball of the first point impacts with the table while the table is going up,
and the second one impacts with the table while the table is going down.
In this case, both movements start close, but the result(the different impacts, times, velocities) is not so close.





\section{References}

The references are shown following alphabetical order:\\

[1]Beukers, F. Dictaat "Modellen en simulaties"

[2]Rudin, W. (1976). Principles of mathematical analysis. International Series in Pure and Applied Mathematics (Third ed.). New York: McGraw-Hill Book Co. pp. 221–223.

[3]Sleijpen G. Slides chapter 3 'Modellen en simulatie'.





\end{document}
